\documentclass[11pt]{article}

\usepackage{amsmath, amssymb}
\usepackage[a4paper, total={7in, 10in}]{geometry}
\usepackage{listings}
\usepackage{url}

%if you really want, you can customise the code syntax highlighting
\lstset{ frameround=tttt, breaklines=false, basicstyle=\ttfamily, frame=single }

\begin{document}

\section{this is a section heading}
\subsection{this is a subsection heading}
\subsubsection{this is a subsubsection heading}

So you want to try some tex?

Inline math is like this $TRUE = (\lambda xy.x)$

Or they can go on their own line like this $$FALSE = (\lambda xy.y)$$

You can get different sized parentheses like this:
$$ \Bigg( \bigg( \Big( \big( ( ) \big) \Big) \bigg) \Bigg) $$

If you wanted to do the full notation (don't) the application is $\cdot$, e.g. $\Big(\lambda x.\big(\lambda y.(y\cdot x)\big)\Big)$

You can do subscripts in math $x_1, x_2, x_{12}$, and superscripts $x^1, x^2, x^{12}$, or both $\Sigma_{bar}^{foo}$

Some fancy symbols: $\rightarrow, \rightarrow_\beta, \twoheadrightarrow_\beta, \equiv, =_\beta...$

Some not so fancy symbols: $\{a, b, c\}$ ... (we had to escape the brace with a backslash)

Later in the course you might want some these too $\land \lor \Rightarrow \Leftrightarrow \vdash$

In general, if you need to find a symbol go look at \url{http://detexify.kirelabs.org/classify.html}

Often you want to line up formulas:

\begin{align*}
  & F\ \{a,b,c,d\} \\
  =\ & (Y\ H)\ \{a,b,c,d\} \\
  =\ & H\ (Y\ H)\ \{a,b,c,d\} \tag{Y Combinator} \\
  =\ & \Big(\lambda fa.\big(ISNIL\ (TAIL\ a)\big)\ (HEAD\ a)\ \big(f\ (TAIL\ a)\big)\Big)\ (Y\ H)\ \{a,b,c,d\}   \\
  =\ & \Big(\lambda a.\big(ISNIL\ (TAIL\ a)\big)\ (HEAD\ a)\ \big((Y\ H)\ \ (TAIL\ a)\big)\Big)\ \{a,b,c,d\}    \tag{$\beta$-reduction}\\
  =\ & ...
\end{align*}

It aligns the \& characters. Don't forget the double backslashes to end each line.

I'll admit, manually adding all those spaces in mathmode is pretty annoying. I expect there's a much better way to typeset stuff.

% this is a comment

If you don't
put a blank

line between
lines
with text
then
there
isn't
a
gap

%\newpage %uncomment this for a page break

\begin{itemize}
  \item bullet
  \item points
\end{itemize}

\begin{enumerate}
  \item numbered
  \item points
\end{enumerate}

\begin{tabular}{c|c|l|||r|}
  tables & are & nice & once \\ \hline
  you & & & get \\
  the & hang & & of \\ \hline \hline
  them & & & \\ \hline
\end{tabular}

Here's some \texttt{truetype font} for where you don't need \emph{fancy} \textbf{highlighting}.

\begin{lstlisting}[language=Lisp]
;; here's some source code, for when you do
;; it's verbatim -- you get exactly what you write, see: $\equiv_\beta$
(defun insert (tree x)
;; inserts x to the binary search tree
)
(defun list-to-tree (mylist &optional tree)
;; note: the second argument 'tree' will be nil by default
;; inserts every element of the list 'mylist' into the tree
)
(defun inorder (tree)
;; list giving the inorder traversal of the tree
)
\end{lstlisting}

OK, that wasnt very fancy. You can make it prettier if you really want, e.g.

\url{https://www.overleaf.com/latex/examples/syntax-highlighting-in-latex-with-the-listings-package/jxnppmxxvsvk#.W4J-guj7QuV}


\end{document}